\documentclass[11pt]{article}
\usepackage[top=2cm,bottom=2cm]{geometry}
\usepackage[T1]{fontenc}
\usepackage{url}
\usepackage{hyperref}
\usepackage{xcolor}
% 
\usepackage{amsthm}
\usepackage{amsmath}
\usepackage{amssymb}
\usepackage{ifthen}
\usepackage{mathpartir}
\usepackage{mathtools}
\definecolor{pblue}{rgb}{0.13,0.13,1}
\definecolor{pgreen}{rgb}{0,0.5,0}
\definecolor{pred}{rgb}{0.9,0,0}
\definecolor{pgrey}{rgb}{0.46,0.45,0.48}
\definecolor{beige}{rgb}{0.96, 0.96, 0.86}
\definecolor{cosmiclatte}{rgb}{1.0, 0.97, 0.91}
\definecolor{cream}{rgb}{1.0, 0.99, 0.82}
\definecolor{ggreen}{rgb}{0.0, 0.5, 0.0}
\definecolor{rred}{rgb}{1.0, 0.13, 0.32}
\definecolor{bblue}{rgb}{0.0, 0.5, 1.0}
\definecolor{upmaroon}{rgb}{0.0, 0.2, 0.67}

\renewcommand{\emph}[1]{\textbf{#1}}

\DeclareMathOperator{\defi}{\triangleq}

\DeclareMathOperator{\concat}{\text{\texttt{++}}}

\newcommand{\menquote}[1]{{\text{``} #1 \text{''}}} 


\newcommand{\kw}[1]{\mathtt{#1}} 
\newcommand{\var}[1]{{\color{bblue}#1}}
\newcommand{\tvar}[1]{{\color{upmaroon}#1}}

\newcommand{\naturals}{\mathbb{N}}
\newcommand{\bnfsep}{\; | \;}
\newcommand{\qst}{\, : \,}
\newcommand{\msubs}[3]{#1[#2 /{#3}]}
\newcommand{\subs}[3]{\msubs{#1}{#2}{\var{#3}}}
\newcommand{\tsubs}[3]{\msubs{#1}{#2}{\tvar{#3}}}


\newcommand{\dom}[1]{\mathit{dom}(#1)}
\newcommand{\QUAL}{{\mathsf{Qua}}}
\newcommand{\TYPES}{\tvar{\mathsf{Typ}}}
\newcommand{\PTYPES}{{\mathsf{PreTyp}}}
\newcommand{\EXPS}{\var{\mathsf{Exp}}}
\newcommand{\AEXPS}{\mathsf{AExp}}
\newcommand{\BEXPS}{\mathsf{BExp}}
\newcommand{\STMTS}{\mathsf{Stmt}}
\newcommand{\VARS}{\mathcal{V}}


\newcommand{\lgge}[1]{\textbf{#1}}
%
\newcommand{\lggeE}{\lgge{E}}
\newcommand{\lggeEF}{\lgge{EF}}
\newcommand{\lggeT}{\lgge{T}}
\newcommand{\lggeW}{\lgge{W}}
\newcommand{\lggeF}{\lgge{F}}
\newcommand{\lggeL}{\lgge{L}}
\newcommand{\lggeX}{\lgge{EX}}
%



\newcommand{\binop}[3]{\kw{#1}(#2; #3)}
\newcommand{\unop}[2]{\kw{#1}(#2)}


\newcommand{\litstr}[1]{\menquote{#1}}

%%%% LANGUAGE E
\newcommand{\enum}[1]{\kw{num}\ifthenelse{\equal{#1}{}}{}{[#1]}}
\newcommand{\estr}[1]{\kw{str}\ifthenelse{\equal{#1}{}}{}{[#1]}}
\newcommand{\ebool}[1]{\kw{bool}\ifthenelse{\equal{#1}{}}{}{[#1]}}
\newcommand{\elbool}[2]{\kw{bool}_{#2}\ifthenelse{\equal{#1}{}}{}{[#1]}}
\newcommand{\ktrue}{\kw{true}}
\newcommand{\kfalse}{\kw{false}}


\newcommand{\eplus}[2]{\binop{plus}{#1}{#2}}
\newcommand{\etimes}[2]{\binop{times}{#1}{#2}}
\newcommand{\ecat}[2]{\binop{cat}{#1}{#2}}
\newcommand{\elen}[1]{\unop{len}{#1}}
\newcommand{\elet}[3]{\kw{let}(#1; \var{#2} . \ #3)}
\newcommand{\ediv}[2]{\binop{div}{#1}{#2}}
\newcommand{\eerror}{\kw{{\color{rred} error}}}
\newcommand{\eshow}[1]{\kw{show}(#1)}

\newcommand{\klet}{\kw{let}}
\newcommand{\kbe}{\kw{be}}
\newcommand{\kin}{\kw{in}}
\newcommand{\clet}[3]{\klet \ \var{#2} \  \kbe \ #1  \ \kin \  #3}
\newcommand{\cleta}[2]{\klet \ \var{#2} \  \kbe \ #1}
\newcommand{\cletb}[1]{\kin \  #1}



\newcommand{\grulename}[1]{\textsc{#1}}
\newcommand{\tyrule}[3]{\inferrule*[Left=#1]{#2}{#3}}
\newcommand{\tyrulename}[2]{#1-ty-#2}
\newcommand{\etyrulename}[1]{\tyrulename{\lggeE}{{#1}}}
\newcommand{\eftyrulename}[1]{\tyrulename{\lggeEF}{{#1}}}
\newcommand{\ttyrulename}[1]{\tyrulename{\lggeT}{{#1}}}
\newcommand{\ftyrulename}[1]{\tyrulename{\lggeF}{{#1}}}
\newcommand{\ltyrulename}[1]{\tyrulename{\lggeL}{{#1}}}


\newcommand{\tyjudge}[3]{#1 \vdash #2 : #3}


\newcommand{\valjudge}[1]{#1 \ \mathsf{{\color{ggreen} val}}}
\newcommand{\errjudge}[1]{#1 \ \mathsf{{\color{rred} err}}}


\newcommand{\semrule}[3]{\inferrule*[Left=#1]{#2}{#3}}
\newcommand{\semrulename}[2]{#1-se-#2}
\newcommand{\esemrulename}[1]{\semrulename{\lggeE}{{#1}}}
\newcommand{\efsemrulename}[1]{\semrulename{\lggeEF}{{#1}}}
\newcommand{\tsemrulename}[1]{\semrulename{\lggeT}{{#1}}}
\newcommand{\wsemrulename}[1]{\semrulename{\lggeW}{{#1}}}
\newcommand{\fsemrulename}[1]{\semrulename{\lggeF}{{#1}}}
\newcommand{\lsemrulename}[1]{\semrulename{\lggeL}{{#1}}}

\newcommand{\trans}{\longmapsto}
\newcommand{\transC}{\longmapsto^{\ast}}
\newcommand{\jtrans}[2]{#1 \trans #2}
\newcommand{\jtransC}[2]{#1 \transC #2}



\newcommand{\anorule}[2]{\inferrule*{#1}{#2}}

%%%%%% LANGUAGE EF %%%%%%%%%%%%%%

\newcommand{\tyarr}[2]{\kw{arr}{(#1; #2)}}
\newcommand{\ctyarr}[2]{#1 \rightarrow #2}
\newcommand{\elam}[3]{\kw{lam}{\{#1\}}{(\var{#2}. \ #3)}}
\newcommand{\clam}[3]{\lambda(\var{#2}:#1) . \ (#3)}
\newcommand{\eapp}[2]{\kw{ap}(#1; #2)}
\newcommand{\capp}[2]{(#1) \ (#2)}


%%%%%% LANGUAGE T %%%%%%%%%%%%%%

\newcommand{\tynat}{\kw{nat}}
\newcommand{\ez}{\kw{z}}
\newcommand{\esucc}[1]{\kw{s}(#1)}
\newcommand{\erec}[5]{\kw{rec}\{#1; \var{#2} . \var{#3} . #4\} (#5)}
\newcommand{\crec}[5]{\kw{rec} \ #5 \ \{ \ez \hookrightarrow  #1
  \ | \
  \esucc{\var{#2}} \ \kw{with} \ \var{#3} \hookrightarrow #4 \}}


\newcommand{\eiter}[4]{\kw{iter}\{ #1; \var{#2} . #3 \} (#4)}
\newcommand{\citer}[4]{\kw{iter} \ #4 \ \{ \ez \hookrightarrow  #1
  \ | \
  {\var{#2}} \hookrightarrow #3 \}}


%%%%%%%%%%%%%%%% DATA TYPES

\newcommand{\tyunit}{\kw{unit}}
\newcommand{\typrod}[2]{\kw{prod}(#1;#2)}
\newcommand{\etriv}{\kw{triv}}
\newcommand{\ctriv}{\langle \rangle}
\newcommand{\epair}[2]{\binop{pair}{#1}{#2}}
\newcommand{\cpair}[2]{\langle #1, #2 \rangle}
\newcommand{\eprl}[1]{\kw{pl}(#1)}
\newcommand{\eprr}[1]{\kw{pr}(#1)}
\newcommand{\cprl}[1]{#1 \cdot \kw{l}}
\newcommand{\cprr}[1]{#1 \cdot \kw{r}}



\newcommand{\tyvoid}{\kw{void}}
\newcommand{\tysum}[2]{\kw{sum}(#1;#2)}
\newcommand{\eabort}[2]{\kw{abort}\{#1\}(#2)}
\newcommand{\cabort}[1]{\kw{abort}(#1)}
\newcommand{\einl}[3]{\kw{inl}\{#1;#2\}(#3)}
\newcommand{\einr}[3]{\kw{inr}\{#1;#2\}(#3)}
\newcommand{\cinl}[1]{\kw{l} \cdot #1}
\newcommand{\cinr}[1]{\kw{r} \cdot #1}
\newcommand{\ecase}[5]{\kw{case}(#1; \var{#2} . \ #3 ; \var{#4} . \ #5)}
\newcommand{\ccase}[5]{\kw{case}\ #1 \ \{\kw{l} \cdot \var{#2}  \hookrightarrow #3 \ | \ 
  \kw{r} \cdot \var{#4} \hookrightarrow #5\}}


%%%%%%%%%%%%%%%%%% T LANGUAGE
\newcommand{\bnum}[1]{\overline{#1}}


%%%%%%%%%%%%%%%%%% F LANGUAGE
\newcommand{\tyall}[2]{\kw{all}{(\tvar{#1} . #2)}}
\newcommand{\ctyall}[2]{\forall(\tvar{#1} . #2)}

\newcommand{\eLAM}[2]{\kw{Lam}(\tvar{#1}.#2)}
\newcommand{\cLAM}[2]{\Lambda (\tvar{#1}) . #2}
\newcommand{\eAPP}[2]{\kw{Ap}\{#1\}(#2)}
\newcommand{\cAPP}[2]{#2 \lceil #1 \rceil}


\newcommand{\typeok}{\mathsf{{\color{ggreen} type}}}
\newcommand{\typejudge}[2]{#1 \vdash #2 \ \typeok}

\DeclareMathOperator{\ctxtsep}{;}

\newcommand{\tyFjudge}[4]{#1 \ctxtsep #2 \vdash #3 : #4}


%%%%%%%%%%%%%%%%%%%%%%% L LANGUAGE

\newcommand{\ellam}[4]{\kw{llam}_{#4}{\{#1\}}{(\var{#2}.#3)}}
\newcommand{\cllam}[4]{\lambda_{#4} (\var{#2}:#1) . #3}
\newcommand{\elpair}[3]{\kw{pair}_{#3}({#1; #2)}}
\newcommand{\clpair}[3]{#3 \ \langle #1, #2 \rangle}

\newcommand{\esplit}[4]{\kw{let}(#1; (\var{#2}, \var{#3}) . #4)}

\newcommand{\ksplit}{\kw{let}}
\newcommand{\kas}{\kw{as}}
\newcommand{\csplit}[4]{\ksplit \  \var{#2}, \var{#3} \ \kbe \ #1 \  \kin \  #4}

\newcommand{\unq}{{\color{ggreen}\mathsf{un}}}
\newcommand{\linq}{{\color{rred}\mathsf{lin}}}
\newcommand{\ptype}[2]{#1 \ #2}


\DeclareMathOperator{\ctxtsplit}{\circ}


\newcommand{\unres}[1]{\unq(#1)}
\newcommand{\linear}[1]{\linq(#1)}


%%%%%%%%%%% WHILE LANGUAGE

\newcommand{\eleq}[2]{\binop{leq}{#1}{#2}}

\newcommand{\sskip}{\kw{skip}}
\newcommand{\csseq}[2]{#1; #2}
\newcommand{\sseq}[2]{\binop{seq}{#1}{#2}}
\newcommand{\site}[3]{\kw{if}(#1; #2 ;#3)}
\newcommand{\csite}[3]{\kw{if}\ #1 \ \kw{then} \ #2 \ \kw{else} \ #3}
\newcommand{\swhile}[2]{\binop{while}{#1}{#2}}
\newcommand{\cwhile}[2]{\kw{while} \ #1 \ \kw{do} \ #2}

\newcommand{\sset}[2]{\kw{set}(\var{#1};#2)}
\newcommand{\cset}[2]{\var{#1} \leftarrow #2}


\newcommand{\upmap}[3]{#1[\var{#2} \mapsto #3]}

\newcommand{\wtyrulename}[1]{\tyrulename{\lggeW}{{#1}}}
\newcommand{\wconf}[2]{\langle #1, #2 \rangle}


\newcommand{\wejudge}[3]{#1 \vdash #2 : #3}
\newcommand{\wsjudge}[2]{#1 \vdash #2 \ {\color{ggreen} \mathsf{ok}}}


\newcommand{\bsjudge}[2]{#1 \Downarrow #2}


%%%%%%%%%%%%% UNTYPED LAMBDA

\newcommand{\eulam}[2]{\kw{lam}{(\var{#1}.#2)}}
\newcommand{\culam}[2]{\lambda \var{#1} . #2}





%%% Local Variables:
%%% mode: latex
%%% TeX-master: "main"
%%% End:

% 
\begin{document}

\title{CO663 \LaTeX\ template}

\date{\vspace{-5ex}}
\maketitle

% \pagenumbering{gobble}

\newcommand{\answerbox}[1]{\framebox{\parbox[c][#1]{\textwidth}{
      \color{white}{h}% 
    }}}



\section{Syntax and typing rules (for language $\lggeE$)}


\subsection{Syntax}

There is \LaTeX\ macro for each construct, see the sources of the
table below.

  \[
  \begin{array}{llclll}
    \TYPES & \tau & \Coloneqq & \enum{}  & \enum{} & \text{numbers}
    \\
           &&& \estr{} & \estr{} & \text{strings}
    \\
    \\ 
    % 
    \EXPS & e & \Coloneqq  & \var{x} & \var{x} & \text{variable}
    \\
           &&& \enum{n} & n & \text{numeral}
    \\
           &&& \estr{s} & \litstr{s} & \text{literal}
    \\
           &&& \eplus{e_1}{e_2} & e_1 {+} e_2 & \text{addition}
    \\
           &&& \etimes{e_1}{e_2} & e_1 {*} e_2 & \text{multiplication}
    \\
           &&& \ecat{e_1}{e_2} & e_1 \concat e_2 & \text{concatenate}
    \\
           &&& \elen{e} & \lvert e \rvert & \text{length}
    \\
           &&& \elet{e_1}{x}{e_2} & \clet{e_1}{x}{e_2} & \text{definition}
  \end{array}
  \]
\subsection{Typing rules}


The macro for typing judgement is called \texttt{tyjudge}, it takes
three arguments: the environment $\Gamma$, the expression being typed
$e$, and its type $\tau$. 
%
The macro to type-set typing rules (and trees) is called
\texttt{tyrule}, it also takes three arguments: the name of the rule,
the premises (separated by $\backslash\backslash$ or
$\backslash$\texttt{and}), and the conclusion.

  \[
  \tyrule
  {\etyrulename{var}}
  {\,}
  {\tyjudge{\Gamma_1, \var{x}: \tau, \Gamma_2}{\var{x}}{\tau} }
  \]
  \[
  \tyrule
  {\etyrulename{str}}
  {\,}
  {\tyjudge{\Gamma}{\estr{s}}{\estr{}}}
  \qquad\qquad\qquad
  \tyrule
  {\etyrulename{num}}
  {\,}
  {\tyjudge{\Gamma}{\enum{n}}{\enum{}}}
  % \end{array}
  \]
  \[
  \tyrule
  {\etyrulename{plus}}
  {
    \tyjudge{\Gamma}{e_1}{\enum{}}
    \and
    \tyjudge{\Gamma}{e_2}{\enum{}}
  }
  {\tyjudge{\Gamma}{\eplus{e_1}{e_2}}{\enum{}}}
  \qquad\qquad\qquad
  \tyrule
  {\etyrulename{times}}
  {
    \tyjudge{\Gamma}{e_1}{\enum{}}
    \and
    \tyjudge{\Gamma}{e_2}{\enum{}}
  }
  {\tyjudge{\Gamma}{\etimes{e_1}{e_2}}{\enum{}}}
  % 
  \]
  \[
  \tyrule
  {\etyrulename{cat}}
  {
    \tyjudge{\Gamma}{e_1}{\estr{}}
    \and
    \tyjudge{\Gamma}{e_2}{\estr{}}
  }
  {\tyjudge{\Gamma}{\ecat{e_1}{e_2}}{\estr{}}}
  \qquad\qquad\qquad
  \tyrule
  {\etyrulename{len}}
  {
    \tyjudge{\Gamma}{e}{\estr{}}
  }
  {\tyjudge{\Gamma}{\elen{e}}{\enum{}}}
  \]
  \[
  \tyrule
  {\etyrulename{let}}
  {
    \tyjudge{\Gamma}{e_1}{\tau_1}
    \and
    \tyjudge{\Gamma, \var{x} : \tau_1 }{e_2}{\tau}
  }
  {\tyjudge{\Gamma}{\elet{e_1}{x}{e_2}}{\tau}}
  \]

\section{Typing derivation}

Type-setting the typing derivation of
$\elet{\enum{42}}{z}{\eplus{\var{z}}{1}}$ (which has type $\enum{}$):



  
  {\small
  \[
  \tyrule{{r1}}
  {
    \tyrule{{r2}}
    {\,}
    {
      \tyjudge{\emptyset}
      {\enum{42}}
      {\enum{}}
    }
    \\
    \qquad   \qquad
    \tyrule{{r3}}
    {
      \tyrule{{r2}}
      {\,}
      {
        \tyjudge{\var{z} : \enum{}}
        {\var{z}}
        {\enum{}}
      }
      \\
      \qquad   \qquad
      \tyrule{{r4}}
      {\,}
      {
        \tyjudge{\var{z} : \enum{}}
        {\enum{1}}
        {\enum{}}
      }
    }
    { 
      \tyjudge{\var{z} : \enum{}}
      {\eplus{\var{z}}{\enum{1}}}
      {\enum{}}
    }
  }
  {
    \tyjudge{\emptyset}
    {\elet{\enum{42}}{z}{\eplus{\var{z}}{\enum{1}}}}
    {\enum{}}
  }
  \]
}


\section{Evaluation}

The macro of evaluation steps is called \texttt{jtrans}, it takes two
arguments: the initial expression and its result.
% 
The macro to typeset evaluation rules and trees is called
\texttt{semrule}, it works like \texttt{tyrule}.
%
To evaluate $\elet{\enum{42}}{z}{\eplus{\var{z}}{1}}$, we proceed in
several steps:

\begin{enumerate}
\item First apply the rule for \texttt{let}:
  \[
  \semrule
  {r5}
  {\,}
  {
    \jtrans
    {\elet{\enum{42}}{z}{\eplus{\var{z}}{\enum{1}}}}
    {
      {\eplus{\enum{42}}{\enum{1}}}
    }
  }
  \]
\item Then we apply the rule for \texttt{plus}:
  \[
  \semrule
  {r6}
  {\,}
  {\jtrans
    {\eplus{\enum{42}}{\enum{1}}}
    {\enum{43}}
  }
  \]
\end{enumerate}



\end{document}


%%% Local Variables:
%%% mode: latex
%%% TeX-master: t
%%% End:
